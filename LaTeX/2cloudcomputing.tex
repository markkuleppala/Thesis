\chapter{Virtualization and Cloud Computing}
\label{chapter:cloudcomputing}

xCloud computing is a vital enabler in the field of Information Technologies. The ubiquitous and connected computing provided allows higher usage of resources and access from anywhere via internet, thus improving performance, reducing costs, and allowing easy access to hosted applications.

A cloud is a pool of virtualized resources across the internet, that follows pay-per-use model and can be dynamically reconfigured to satisfy the user requests by provisioning virtual machines. Cloud computing is a service model for IT provisioning, often based on virtualization and distributed computing technologies \cite{Lombardi2011}. Cloud is typically a centralized server, such as, Microsoft Azure or Red Hat OpenShift. These cloud platforms provide users with scalable resources, very high availability and fast connections. Cloud platforms can be public, such as the before-mentioned ones, or a private cloud serving only selected customers. Private clouds are usually deployed on-premises by businesses with requirements for additional control and privacy over the network. \cite{MicrosoftAzure}

Community cloud

Hybrid cloud is a combination of both, made up of on-premises infrastructure, private cloud services, and a public cloud. The hybrid approach brings enterprise with agility to use the best suitable solution for each occasions. \cite{NetApp}


\section{History of cloud computing}

The concept of virtualization technologies dates back to 

\section{Far Edge Cloud}

Benefits of far edge
    - Latency
    - Allowing thin clients / offloading task to edge
    - Price and effect to environment
    - Security
    - Centralized repairment / upgrading
    - Less power consumption

\section{Virtualization}

Virtualization takes advantage of the cloud computing platform. That is, software can be installed allowing multiple instances of virtual servers to be used \cite{Velte2009}.

Run different guest OS in different host OS

With the help of virtualization a single physical server can run a dozen of separate virtual servers.

\subsection{Prerequisites of virtualization}

\subsection{Different kinds of virtualization}

\subsection{Security concerns}



Background of virtualization needs \\
MECs / Far Edge Clouds \\
    - What is it? Why it is important? How it is achieved? Where? \\
Containers \\
Security concerns of containers \\
Why VMs? \\

%You can translate your latex file to rtf with the \texttt{latex2rtf} command in the
%kosh.aalto.fi shell server. Then, the line breaks
%will not be problems for the proofreader of Word.

%Note also that if you have a section or a subsection, you have to have
%at least two of them, or otherwise the section or subsection title is
%unnecessary. Same with the paragraphs: you should not have sections
%with only one paragraph, and single sentence paragraph.

%Aalto library has a comprehensive citation guide ~\cite{bibinstructions}.