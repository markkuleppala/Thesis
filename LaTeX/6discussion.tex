\chapter{Discussion}
\label{chapter:discussion}

Kata Containers' performance lacks behind the runC's performance in bandwidth and completion latency. Due to the caching of the hypervisor, no precise overhead number can be drawn from these tests for read operations. For write operations, the overhead is around 33-40\% for 4 kB, the default Unix block size. Kata Containers can add up to 500 ms to completion latency for the 50th percentile of task completion. Kata Containers' performance depends on the underlying system, such as memory-based storage volume, where the overhead is negligible. The bandwidth degradation grows proportionally larger in disk-based volumes than in memory-based volumes.

The provided performance could be enough for non-performance critical applications, which require no real-time data processing and transmission, such as logging. However, applications with strict requirements for latency and bandwidth might find the added overhead more harmful than the increased security. Luckily, the Kubernetes environment can run multiple runtimes simultaneously, allowing the assignment of the most suitable runtime for each situation. So, for example, trusted and performance-critical applications could be assigned with runC, and untrusted workloads with no strict requirements could harness Kata Containers.

Kata Containers offers a wide variety of features that support the most critical telco and Far Edge Cloud environment needs. In addition, the architecture of micro-VMs allows even more robust deployments that traditional Kubernetes deployments do not offer with a shared kernel.

\section{Future work}

This thesis conducted tests on a single node cluster. In future work, it would be interesting to research the performance more extensively on a multi-node cluster and examine the I/O performance properties of data transfer between containers located in different pods or I/O operations' performance between containers in a pod. The container orchestrator in this thesis is K3s, a lightweight version of Kubernetes. However, the majority of production environments are running on the Kubernetes. Therefore, future work could use Kubernetes as a container orchestrator to align with the default deployment of cloud computing with the latest version of Kata Containers.

Due to incompatibility issues, the Firecracker runtime and CPU isolation plugin were dropped from the scope. Firecracker is promising container runtime backed by an active community. Evaluating Firecracker's performance against other runtimes introduced in this thesis would be highly beneficial when selecting the best performing runtime. CPU isolation dedicates a core for workload, eliminating the undesired interruptions to test processes from the system. Enabling this feature would increase the credibility of the test results with the guarantee of non-interrupted test runs.

The inclusion of hypervisor to the Kata Containers architecture adds a cache to the environment. Unfortunately, this cache falsely improved the performance of Kata Containers, disabling the comparability against runC or bare-metal tests results. For future work, it would be beneficial to minimize the effect of caching in the tests to allow performance evaluation of read tests.

In addition to I/O performance, Kata Containers affects overall performance on other factors, such as adding memory and CPU overhead which should also be measured. In addition, it would be important to locate the origin of performance overhead and possible solutions to mitigate the performance degradation.

\section{Conclusion}

Telco infrastructure hosting various third-party applications on top of Far Edge Cloud deployment requires a sophisticated isolation mechanism to secure users and data. Kata Containers offers a high potential solution to provide containerized environments with an extra layer of security against security threats, such as container escapes. Furthermore, the features of Kata Containers support Far Edge Cloud applications with the most critical requirements, such as SR-IOV, Multus, and hardware acceleration. From the experiments performed, it is evident that Kata Containers adds its performance overhead to disk-based I/O operations. However, Kata Containers' performance is constantly optimized, and it is suitable for non-performance critical applications in its current state.