\chapter{Introduction}
\label{chapter:intro}

During the past years, technology companies have widely adopted hardware virtualization technologies such as Docker. This virtualization allows a dynamic sharing of computing resources between applications running on the host system. The virtualization brings a wide range of improvements to running the application on bare-metal. For example, wrapped Docker container applications are highly mobile standalone applications easily deployed on any environment supporting containers with minimal overhead. Also, the platforms providing infrastructure for the virtualized application enables orchestration features such as replication and automatic deployment. These features could improve the infrastructure's performance, resource allocation, and uptime by automatically handling application errors or scaling or resources. These virtualization technologies are leveraged in various parts of the mobile broadband network to enhance service quality.

Multi-access Edge Computing (MEC) greatly benefits from the adaption of virtualization technologies. MEC is a crucial technology enabler for Fifth Generation Technology Standard for Broadband Cellular Networks (5G) technologies to serve users with low-latency and high-throughput mobile network connections. This shift enables Mobile Network Operators (MNOs) to host various applications in physical proximity to the end-users. Far Edge Cloud takes computing further to the edge of the network by closing the physical distance between user and server to be at a maximum of 40 kilometers \cite{AirFrameOpenEdgeServer}. The MEC close to the user in Far Edge Cloud increases Quality-of-Experience; meanwhile, end-users enjoy low-latency, on-premise data computing, and additional location services. The benefits of MEC for MNOs include additional revenues from hosting third-party applications on the network.

These MEC platforms are currently leveraging containers for orchestrating workloads. Container architecture shares a host kernel with the third-party applications. Recently, it has been discovered that container architecture flaws have led to malicious applications escaping the container \cite{CVE-2020-14386}\cite{CVE-2019-5736}. As all containers hosted by the system share the host kernel, these malicious applications can compromise other containers running on the same system. Leveraging this exploit might lead to a breach of information from the otherwise secure applications and exposes third-party applications to each other via lateral attack vectors.

Kata Containers \cite{KataContainers} has been introduced as a promising solution to add an extra layer of isolation to the applications to secure, for example, the MEC platforms. This extra layer is achieved by running all containers inside a lightweight virtual machine (VM). This VM creates an application-specific mini-kernel and hypervisor to isolate containers from the host and other untrusted workloads. This extra layer of security provided by Kata Containers minimizes the attack surface between applications. However, more complex architecture and added components in the environment add overhead to the application's performance.

\section{Problem statement}
\label{section:intro_problemstatement}

The main research question of this thesis is to evaluate if Kata Containers is suitable container runtime for Kubernetes orchestrated Far Edge Cloud environment. The I/O performance of Kata Containers is compared against bare-metal, where the application runs directly on top of the OS, and the current default of Kubernetes, Docker runtime runC. This thesis investigates the performance impact of an additional isolation layer to applications with Kata Containers as runtime of Far Edge Cloud setting.

Far Edge Clouds support end-user devices with applications, some of which require sophisticated features such as accelerators, direct access to network interface controllers (NICs), and support for multiple parallel network interfaces. The second research question is to map out the possible gaps with the supported features of Kata Containers in the Far Edge Cloud setting and explore the possible new features the Kata Containers' architecture adds.

\section{Scope and Methodology}
\label{section:intro_scopemethodology}

This thesis focuses on telecommunications (telco) applications and the needs these applications have. It is essential to focus on the I/O performance, compatibility, and possible gaps Kata Containers might add. The majority of the applications are based on Linux operating systems, limiting the performance evaluation scope to Linux operating systems (OS). Kata Containers is considered the potential solution for the runtime, as it is currently savoring the broadest adoption and range of features \cite{Flauzac2020}. Kubernetes is chosen as the container orchestrator for the same reason.

The first part evaluates Kata Containers concerning the Far Edge Cloud environment via literature review. In the second part, Kata Containers is deployed on an environment resembling Far Edge Cloud environment. Finally, the performance of various hypervisors and storage volumes are evaluated with disk and memory-based I/O performance tests.

\section{Results}
\label{section:intro_results}

Kata Containers is evaluated against the requirements of Far Edge Cloud applications. It is found, Kata Containers supports the most critical features such as core isolation and hardware acceleration. These features enable high-performing applications to unlock their potential with dedicated computing resources. In the I/O performance evaluation, Kata Containers adds only a minor overhead for I/O bandwidth and completion latency with memory-based volume. In contrast, with disk-based volumes, the overhead grows greater, resulting in up to 40\% I/O bandwidth degradation and over 200 ms increase to completion latency against runC, the native runtime of Kubernetes.

\section{Structure of the Thesis}
\label{section:intro_structure}

Chapter \ref{chapter:cloudcomputing} describes virtualization, cloud computing, and Far Edge Cloud environment. This chapter also discusses the requirements of Edge Cloud and the security concerns of cloud computing. Chapter \ref{chapter:katacontainers} focuses on the secure container runtimes and Kata Containers architecture. The Far Edge Cloud environment and Kata Containers are combined in Chapter \ref{chapter:implementation} which implements the runtime in the Far Edge Cloud context and discusses the requirements to the environment and applications. The performance is evaluated thoroughly in Chapter \ref{chapter:evaluation} and the performance is analyzed based on the acquired results. The last Chapter \ref{chapter:discussion} focuses on the compatibility and possible gaps in provided features, discusses the test results and future work, and concludes the thesis.