\chapter{Introduction}
\label{chapter:intro}

Introduction tells the motivation, scope, goal and the outcome of the
work. Anyone should be able to understand it. The preferred order of
writing your master's thesis is about the same as the outline of the
thesis: you first discover your problem and write about that, then you
find out what methods you should use and write about that.  Then you
do your implementation, and document that, and so on.  However, the
abstract and introduction are often easiest to write last.  This is
because these really cover the entire thesis, and there is no way you
could know what to put in your abstract before you have actually done
your implementation and evaluation. This means that you have to
rewrite them in the end of your work.

Read the information from the university master's thesis
pages~\cite{ThesisInstructions} before starting the thesis.  You
should also go through the thesis grading
instructions~\cite{ThesisGrading} together with your advisor and/or
supervisor in the beginning of your work.

The introduction in itself is rarely very long; two to five pages
often suffice. It usually has two subsections with titles Problem
statement and Structure of the Thesis, as follows next.

\section{Problem statement}
\label{section:problem} 

MEC applications requiring low latency \\
Security issues of containers \\
Slowness/lack of orchestration in VMs \\
KC as a solution \\


\section{Structure of the Thesis}
\label{section:structure} 



