\chapter{Introduction}
\label{chapter:intro}

During the past years, technology companies have widely adopted virtualization technologies such as Docker. The virtualization brings a wide range of improvements to running the application on bare-metal. For example, wrapped Docker container applications are highly mobile standalone applications easily deployed on any environment supporting containers with minimal overhead. Also, the virtualized application supports orchestration features such as replication and automatic deployment. These virtualization technologies are leveraged in various parts of the mobile broadband network.

Multi-access Edge Computing (MEC) greatly benefits from the adaption of virtualization technologies. MEC is a crucial technology enabler for Fifth Generation Technology Standard for Broadband Cellular Networks (5G) technologies to serve users with low-latency and high-throughput mobile network connections. This shift enables Mobile Network Operators (MNOs) to host various applications in physical proximity to the end-users. Far Edge Cloud takes computing further to the edge of the network by closing the physical distance between user and server to be at a maximum of 40 kilometers \cite{AirFrameOpenEdgeServer}. The MEC close to the user in Far Edge increases Quality-of-experience; meanwhile, end-users enjoy low-latency, on-premise data computing, and additional location services. The benefits of MEC for MNOs include additional revenues due to hosting third-party applications on the network.

However, recently, it has been discovered that container architecture flaws have led to malicious applications escaping the container. As all containers hosted by the system share the host kernel, these malicious applications can compromise the other containers running on the same system. Leveraging this exploit might lead to a breach of information from the otherwise secure applications and exposes third-party applications to each other via lateral vectors.

Kata Containers \cite{KataContainers} has been introduced as a promising solution to add an extra layer of isolation to the applications to secure the MEC platforms. This extra layer is achieved by running all containers inside a lightweight virtual machine (VM). This VM creates an application-specific mini-kernel and hypervisor to minimize the attack surface between applications, which adds overhead to the application, as demonstrated in Chapter \ref{chapter:evaluation}.

\section{Problem statement}
\label{section:intro_problemstatement}

The main research question of this thesis is to evaluate Kata Containers as a secure runtime in Kubernetes orchestrated Far Edge Cloud environment. The performance of Kata Containers is compared against bare-metal, where the application runs directly on top of the OS, and the current default of Kubernetes, Docker runtime. This thesis investigates the performance impact of an additional isolation layer to applications with Kata Containers as runtime of Far Edge Cloud setting.

Far Edge Clouds support end-user devices with applications, some of which require sophisticated features such as accelerators, direct access to network interface controllers (NICs), and support for multiple parallel network interfaces. The second research question is to map out the possible gaps with the supported features of Kata Containers in the Far Edge Cloud setting explore the new features the Kata Containers' architecture adds.

\section{Scope and Methodology}
\label{section:intro_scopemethodology}

This thesis focuses on telecommunications (telco) applications and the needs these applications have. It is essential to focus on the performance, compatibility, and possible gaps Kata Containers might add. The majority of the applications are based on Linux operating systems, limiting the performance evaluation scope to Linux operating systems (OS). Kata Containers is considered the potential solution for the runtime, as it is currently savoring the broadest adoption and range of features \cite{Flauzac2020}. Kubernetes is chosen as the container orchestrator for the same reason.

The first part evaluates Kata Containers concerning the Far Edge Cloud environment via literature review. In the second part, Kata Containers is deployed on an environment resembling Far Edge Cloud environment. The performance of various hypervisors and storage volumes are evaluated with I/O performance tests.

\section{Results}
\label{section:intro_results}

\textcolor{red}{Missing a piece here!}

\section{Structure of the Thesis}
\label{section:intro_structure}

Chapter \ref{chapter:cloudcomputing} describes virtualization, cloud computing, and Far Edge Cloud environment. This chapter also discusses the requirements of Edge Cloud and the security concerns of Cloud Computing. Chapter \ref{chapter:katacontainers} focuses on the secure runtimes and Kata Containers architecture. The environment and Kata Containers are combined in a more practical Chapter \ref{chapter:implementation} which implements the runtime in the Far Edge Cloud context and discusses the requirements to the environment and applications. The performance is evaluated thoroughly in Chapter \ref{chapter:evaluation} and the performance is analyzed based on the acquired results. The last Chapter \ref{chapter:discussion} focuses on the gaps in provided features, discusses the possible improvement factors, and concludes the thesis.