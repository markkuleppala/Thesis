% Abstract in English
% ------------------------------------------------------------------
\thesisabstract{english}{
Containerized applications have gained wide popularity recently due to flexibility and great resource allocation. However, the container runtimes provide limited isolation of workloads due to the fact that the Linux kernel is shared by the containers. Telco and radio applications handle sensitive data and need high level of isolation and great security measures to avoid potential exploits of container runtime or kernel vulnerabilities.

Apart from the security requirements, the challenge for virtualizing radio applications is the specific requirements for container runtime features. These features include for example the use of accelerators, direct access to NICs, and support for multiple interfaces which might require elevated access. Operating containers with elevated privileges is considered risky, as the application might escape the container.

This thesis explores Kata Containers as a potential solution, as it provides runtime utilizing lightweight virtual machine run on top of hypervisor. The solution is studied by developing an environment resembling a radio application and evaluating the performance.

In this thesis it was found out, that Kata Containers provides orchestration and performance of container with the isolation of virtual machines. The performance was XXX and support applications even with highest latency requirements when developed in Far Edge Cloud.
}

% Abstract in Finnish
% ------------------------------------------------------------------
\thesisabstract{finnish}{
}