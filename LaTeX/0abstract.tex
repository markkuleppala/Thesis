% Abstract in English
% ------------------------------------------------------------------
\thesisabstract{english}{
Containerized applications have gained wide popularity recently due to flexibility and great resource allocation. However, the container runtimes provide limited isolation of workloads since the containers share the Linux kernel. Telco and radio applications handle sensitive data and need a high level of isolation and extraordinary security measures to avoid potential exploits of container runtime or kernel vulnerabilities.

Apart from the security requirements, the challenge for virtualizing radio applications is the specific requirements for container runtime features. These features include, for example, the use of accelerators, direct access to NICs, and support for multiple interfaces, which might require elevated access. Operating containers with elevated privileges are considered risky, as the application might escape the container.

This thesis explores Kata Containers as a potential solution, as it provides runtime utilizing lightweight virtual machines running on top of a hypervisor. The solution is studied by developing an environment resembling a radio application and evaluating the performance.

In this thesis, it was found out that Kata Containers provides orchestration and performance of a container with virtual machines' isolation. The performance was XXX and support applications even with the highest latency requirements when developed in Far Edge Cloud.
}

% Abstract in Finnish
% ------------------------------------------------------------------
\thesisabstract{finnish}{
}