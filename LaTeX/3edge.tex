\chapter{Edge Data Center}
\label{chapter:edge}

Telecommunications computing ... The computing locates in various data centers as described in Figure \ref{fig:AirFrame}. Central data centers are built to massive warehouses that take advantage of centralized maintenance and 

Telco environment
Central data center vs edge cloud in general
    - Efficient capacity vs low latency \& efficient transport


\begin{figure}[ht]
  \begin{center}
    \includegraphics[width=13.5cm]{images/AirFrame.png}
    \caption{Far edge comparison to central data center \cite{AirFrameOpenEdgeServer}}
    \label{fig:AirFrame}
  \end{center}
\end{figure}

\section{Multi-access Edge Computing}

Multi-access Edge Computing (MEC) places compute and storage resources in the Radio Access Network (RAN) improving the delivery of content and applications to end user. MEC improves network efficiency by processing data on the mobile network cell instead of hauling it completely back to regional or central data centers for processing. The data can be processed partially or completely on the MEC node. For example, large public venues, such as stadiums or arenas, are good candidates for MEC, especially where localized venue services are important. In this use case, video created at a sports event or concert is served to on-site consumers from a MEC server, running appropriate applications located on the stadium premises. This video traffic is locally stored, processed and delivered directly to users at the event and does not require backhaul to a centralized core network and to then be returned to the user at the venue. The local processing of data reduces perceived latency on end-user and limits stress on the backhaul network. \cite{Brown2016}

MEC covers two edge entities: aggregated edge and far edge. Aggregated edge locates usually at most 200 to 350 kilometers away from the end user and guarantees 4 to 10 millisecond round trip time (RTT). The RTT is low in comparison to central data center latency, however it is not low enough for mission-critical applications requiring less than 1 ms latency. For example, self-driving cars and factory robots might require ultra-low latency reply from the applications.

\subsection{Far Edge Cloud}

\textcolor{red}{Emphasize the need for security!}

Far edge cloud locates at most 20 to 40 kilometers away from the end user, and thus is able to offer less than one millisecond RTT. The ultra-low latency is critical for applications requiring instant reply from the server. Offloading tasks, computing, and storage to nearby cloud allows thin clients, such as cellphones or dashboards, with limited resources to support applications with resource heavy requirements, such as applying machine learning models to obtained video feed in real-time. Far edge cloud setup is minimal in size, footprint, and power budget. It is flexible to install as it can be located in an office or apartment building with no requirement for extensive server chassis. \cite{AirFrameOpenEdgeServer}

Far edge cloud units are highly localized. A network might consist of hundreds or thousands of these units. The data processed does not always travel further away from the end-user enhancing the security and integrity of the data. The localized data schemes help also service providers, as they are not obliged to apply multiple data protection laws, GDPR for example, at the same time. 

\subsection{Applications}

MEC architecture unlocks potential of new applications, and thus is a new revenue stream for mobile network operators. Some recognized applications harnessing the use of MEC include augmented reality (AR), video analytics, and connected vehicles.

AR enables real environment user experience by combining real and virtual objects existing simultaneously. Recent AR applications have become adaptive in sound and visual components, such as news, TV programs, sports, object recognition, and games. AR applications often demand high computational resources, low latency for reasonable quality-of-experience, and high bandwidth. MEC can empower AR applications by maximizing throughput by offering computational resources nearby the end-user withing low-latency and high throughput channel instead of relying on the core-network. \cite{Abbas2018}

Surveillance cameras traditionally have been streaming data back to the server, which decides how to perform data analysis. Due to the growing number of these cameras, it adds an enormous stress to the network with the constant data streams. In this example, MEC will be beneficial by implementing intelligence to the device by transmitting data only when motion is detected. Also, it can help public safety with detection of traffic jams, accidents or forest fires.

Connected vehicles are facilitated with cellular connection allowing them to communicate with other users on the road. MEC based road communication system could allow two-way communication between vehicles via road side units. For example, one vehicle could warn another vehicles about upcoming road jam or approaching pedestrians. The road side unit could also warn passing cars about dangerous conditions based on the sensors equipped in it. \cite{Abbas2018}

% Comment: If your sentence ends in a capital letter write \@ before the period

% If you do need a normal space after a period (instead of
% the longer sentence separator), use \  (backslash and space) after the
% period. Like so: a.\ first item, b.\ second item.